\documentclass[t]{beamer}

\usetheme{Berkeley}

\usepackage[orientation=landscape, size=a0, scale=1.4]{beamerposter}
\usepackage[absolute,overlay]{textpos}
\usepackage{amsmath}
\usepackage{amsfonts}
\usepackage{graphicx}

\setlength{\TPHorizModule}{1cm}
\setlength{\TPVertModule}{1cm}

\newcommand{\n}{\par \smallskip}
\newcommand{\nn}{\n \par \smallskip}  
\newcommand{\f}{\n \hrule \n}
\newcommand{\s}[1]{\textbf{#1}}
\newcommand{\img}[2]{\begin{figure}[htp]\fcolorbox{black}{white}{\includegraphics[width=#2\textwidth]{#1}}\end{figure}}
\newcommand{\imgnobox}[2]{\begin{figure}[htp]\includegraphics[width=#2\textwidth]{#1}\end{figure}}
\newcommand{\colwidth}{37.1333}

\newcommand{\TODO}{{\color{red}TODO}}
\newcommand{\TODOC}[1]{\TODO{\color{red}\{#1\}}}

\begin{document}

\begin{textblock}{104.8}(3, 2.5)
\begin{block}{\center \Large $T_2$ Shuffling - Dynamic MRI Dimensionality Reduction}
\center \text{\large Siddharth Iyer, Jonathan I.\ Tamir, Michael Lustig} \par \bigskip
Department of Electrical Engineering and Computer Sciences,  University of California at Berkeley, Berkeley, CA
\end{block}
\end{textblock}
\begin{textblock}{8.09}(109.3,2.5)
\begin{block}{\vspace*{-3ex}}
\imgnobox{seal.png}{0.87}
\end{block}
\end{textblock}

{%%%%%%%%%%%%%%%%%%%%%%%%%%%%%%%%%%%%%%%%%%%%%%%%%%%%%%%%%%%%%%%%%%%%%%%%%%%%%%


\setbeamercolor{block body}{bg=white}
%\setbeamercolor{block title}{#3}


% left column
\begin{textblock}{\colwidth}(3,11.5)
  \begin{frame}
  
  %%%%%%%%%%%%%%%%%%%%%%%%%%%%%%%%%%%%%%%%%%%%%%%%%%%%%%%
  % Introduction
  %%%%%%%%%%%%%%%%%%%%%%%%%%%%%%%%%%%%%%%%%%%%%%%%%%%%%%%
  \begin{block}{Introduction}
  \begin{itemize}
  \item MRI is a safe and powerful tool that can be used to
  image both anatomy and function. 
  \item Acquiring 3D MRI over time has several clinical applications:
    \begin{itemize}
    \item Reduce image blur from conventional 3D acquisition.
    \item Visualize signal behavior over time.
    \item Quantify tissue parameters in anatomy.
    \end{itemize}
  \item We aim to find a robust and low-dimensional representation of 
  the dynamic images.
  \end{itemize}
  \end{block}
  
  %%%%%%%%%%%%%%%%%%%%%%%%%%%%%%%%%%%%%%%%%%%%%%%%%%%%%%%
  % Motivation
  %%%%%%%%%%%%%%%%%%%%%%%%%%%%%%%%%%%%%%%%%%%%%%%%%%%%%%%
  \begin{block}{Motivation}
  \begin{itemize}
  \item The image's Fourier transform (called k-space) is sampled over time. 
  \item The samples are grouped into time-consistent k-space bins.
  \imgnobox{acq_over_time.png}{0.9}\n
  \item \underline{Goal:} Clinically feasible scan times by reducing dimensionality with low model error.
  \end{itemize}
  \end{block}
  
  %%%%%%%%%%%%%%%%%%%%%%%%%%%%%%%%%%%%%%%%%%%%%%%%%%%%%%%
  % Background
  %%%%%%%%%%%%%%%%%%%%%%%%%%%%%%%%%%%%%%%%%%%%%%%%%%%%%%%  
  \begin{block}{Background}
  Signal evolution of pixel intensity over time parametrized by tissue
  $T_2$ relaxation.
  \imgnobox{pix_time.png}{0.91}
  Signals evolutions are correlated $\implies$ data captured by a low-dimensional
  subspace.
  \begin{enumerate}
  \item Estimate $T_2$ distribution from tissue in anatomy of interest.
  \item Simulate curves with $T_2$ values drawn from distribution.
  \item Form matrix $X$ with signals as columns arranged in increasing order of $\ell_2$ norm.
  \end{enumerate}
  \end{block}
  \end{frame}
\end{textblock}

% middle column
%\begin{textblock}{\colwidth}(41.6333,11.5)
\begin{textblock}{\colwidth}(41.6333,10.05)
  \begin{frame}
  \begin{block}{\vspace*{-3ex}}
  \imgnobox{T2data.png}{0.9}
  \end{block}
  
  %%%%%%%%%%%%%%%%%%%%%%%%%%%%%%%%%%%%%%%%%%%%%%%%%%%%%%%
  % Methods
  %%%%%%%%%%%%%%%%%%%%%%%%%%%%%%%%%%%%%%%%%%%%%%%%%%%%%%%
  \begin{block}{Methods and Results}
  \begin{itemize}
  \item {\underline{Principal Component Analysis:}}
  Project $X$ onto first 3 principal components, $U$. This solves:
  $$\underset{U, A}{\text{min}} \quad ||X - UA||_F^2$$ 
  Estimated matrix $X' = UA$. Normalized model error of $X_i'$:
  \imgnobox{bad_svd.png}{0.5} 
  \item {\underline{Multilayer Linear Regressor:}}
  Inspired by Neural Networks, this only uses linear transformations.
  \imgnobox{neural_network.png}{0.6}
  \begin{align*}
  U &\leftarrow \theta_2 \\ A &\leftarrow \theta_1 X
  \end{align*}
  This framework is used to solve:
    $$\underset{U, A}{\text{min}} \quad \underset{i}{\text{max}} \quad ||X_i - Ua_i||_2^2$$ 
  \imgnobox{good_nn.png}{0.5}
  \end{itemize}
  \end{block}
  \end{frame}
\end{textblock}

% right column
% \begin{textblock}{\colwidth}(80.2667,11.5)
\begin{textblock}{\colwidth}(80.2667,10.05)
  \begin{frame}
  %%%%%%%%%%%%%%%%%%%%%%%%%%%%%%%%%%%%%%%%%%%%%%%%%%%%%%%
  % Methods continued
  %%%%%%%%%%%%%%%%%%%%%%%%%%%%%%%%%%%%%%%%%%%%%%%%%%%%%%%
  \begin{block}{\vspace*{-3ex}}
  \imgnobox{algo.png}{0.85}
  \begin{itemize}
  \item{\underline{Results}}
  $$\begin{tabular}{|c|c|c|} \hline
   & \text{PCA} & \text{MLR} \\ \hline
  \text{Global Error} & 0.33\%&   2.46\% \\ \hline
  \text{Maximal Error} & 11.1\% & 4.50\% \\ \hline
  \end{tabular}$$
  \end{itemize}
  \end{block}
  
  %%%%%%%%%%%%%%%%%%%%%%%%%%%%%%%%%%%%%%%%%%%%%%%%%%%%%%%
  % Discussion and Conclusion
  %%%%%%%%%%%%%%%%%%%%%%%%%%%%%%%%%%%%%%%%%%%%%%%%%%%%%%%
  \begin{block}{Discussion and Conclusion}
  \begin{itemize}
  \item Our flexible framework enables the evaluation of different cost functions.
  \item We produce a low-dimensional representation
  with lower maximal error at a small cost of global error.
  \end{itemize}
  \end{block} 
  
  %%%%%%%%%%%%%%%%%%%%%%%%%%%%%%%%%%%%%%%%%%%%%%%%%%%%%%%
  % Future Work
  %%%%%%%%%%%%%%%%%%%%%%%%%%%%%%%%%%%%%%%%%%%%%%%%%%%%%%%
  \begin{block}{Future Work}
  \begin{itemize}
  \item \underline{$T_2$ mapping and classification:} Classifying $T_2$ values from signal evolutions.
  \imgnobox{future.png}{0.8}
  \end{itemize}
  \end{block} 
  
  %%%%%%%%%%%%%%%%%%%%%%%%%%%%%%%%%%%%%%%%%%%%%%%%%%%%%%%
  % References
  %%%%%%%%%%%%%%%%%%%%%%%%%%%%%%%%%%%%%%%%%%%%%%%%%%%%%%% 
  \begin{block}{References}
  [1] Tamir JI, Lai P, Uecker M, Lustig M. Reduced blurring in 3D fast spin echo through joint temporal espirit reconstruction. Proc. Intl. Soc. Mag. Reson. Med. 22 2014; p. 0616. \n
  [2] Busse RF, Hariharan H, Vu A, Brittain JH. Fast spin echo sequences with very long echo trains: design of variable refocusing flip angle schedules and generation of clinical t2 contrast. Magn Reson Med 2006; 55:1030–7. 
  \end{block} 
  
  %%%%%%%%%%%%%%%%%%%%%%%%%%%%%%%%%%%%%%%%%%%%%%%%%%%%%%%
  % Acknowledgements
  %%%%%%%%%%%%%%%%%%%%%%%%%%%%%%%%%%%%%%%%%%%%%%%%%%%%%%%
  \begin{block}{Acknowledgements}
  I would like to thank Jonathan Tamir for allowing me to work with him on his project, Prof. Michael Lustig for letting me be a part of his research group and all the MikGroup members for making me feel welcome.
  \end{block}
  \end{frame}
\end{textblock}
}
  
\end{document}


